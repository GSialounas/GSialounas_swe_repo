\documentclass{beamer}
%
% Choose how your presentation looks.
%
% For more themes, color themes and font themes, see:
% http://deic.uab.es/~iblanes/beamer_gallery/index_by_theme.html
%
\mode<presentation>
{
	\usetheme{default}      % or try Darmstadt, Madrid, Warsaw, ...
	\usecolortheme{default} % or try albatross, beaver, crane, ...
	\usefonttheme{default}  % or try serif, structurebold, ...
	\setbeamertemplate{navigation symbols}{}
	\setbeamertemplate{caption}[numbered]
} 
\newcommand{\qp}[1]{\left(#1\right)}
\newcommand{\qpb}[1]{\left(#1\right]}
\newcommand{\qb}[1]{\left[#1\right]}
\usepackage[english]{babel}
\usepackage[utf8]{inputenc}
\usepackage[T1]{fontenc}

\title[ENO/WENO schemes for shallow water]{ENO/WENO schemes for the Shallow Water  Equations}
\author{Georgios Sialounas}
\institute{UoR, ICL, MPE CDT}
\date{22 Oct 2020}

\begin{document}
	
	\begin{frame}
	\titlepage
\end{frame}

% Uncomment these lines for an automatically generated outline.
%\begin{frame}{Outline}
%  \tableofcontents
%\end{frame}

\section{Introduction}

\begin{frame}{Introduction}
We present the ENO/WENO schemes for the shallow water equations in 1D and potentially in 2D:
\begin{equation}
\begin{aligned}
\begin{array}{c}
h_t\\
q_t
\end{array}
\begin{array}{c}
+\\
+
\end{array}
\begin{array}{c}
\qp{q}_x\\
\qp{hv^2+\frac{1}{2}gh^2}_x
\end{array}
\begin{array}{c}
=0\\
=0
\end{array},
\end{aligned}
\end{equation}
where $q:=hv$.
\vskip 1cm

%\begin{block}{Examples}
%	Some examples of commonly used commands and features are included, to help you get started.
%\end{block}

\end{frame}

\section{Numerical Schemes}



\begin{frame}{ENO/WENO schemes}


\end{frame}

\section{Numerical experiments}


\end{document}
